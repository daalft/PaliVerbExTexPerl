\documentclass[11pt]{article}
\usepackage{enumitem}
\usepackage{fontspec,xunicode,xltxtra}
\usepackage{polyglossia}
\setdefaultlanguage{english}
\setotherlanguage{french}
\setotherlanguage{german}
\setotherlanguage{sanskrit}
\setotherlanguage{thai}
\setotherlanguage{sinhala}

\def\seng{\selectlanguage{english}}
\def\sfra{\selectlanguage{french}}
\def\sger{\selectlanguage{german}}
\def\sskt{\selectlanguage{sanskrit}}

\defaultfontfeatures{Mapping=tex-text}
\setromanfont{Gandhari Unicode}
\setsansfont{Gandhari Unicode}
\setmonofont{Courier}
%\newfontfamily\fjpn{MS PMincho} % font Japanese
%\newfontfamily\ftib{Kokonor} % font Tibetan
%\newfontfamily\sanskritfont{Devanagari MT}
%\newfontfamily\thaifont{Ayuthaya}
%\newfontfamily\malayalamfont{Malayalam MN}
%\newfontfamily\sinhalafont{Malithi Web}

\newcommand*\ṛ{r\symbol{"325}}
\newcommand*\Ṛ{R\symbol{"325}}
\newcommand*\ṝ{r\symbol{"304}\symbol{"325}}
\newcommand*\Ṝ{R\symbol{"304}\symbol{"325}}
\newcommand*\ḷ{l\symbol{"325}}
\newcommand*\ḹ{l\symbol{"304}\symbol{"325}}
\newcommand*\Ḷ{L\symbol{"325}}
\newcommand*\Ḹ{L\symbol{"304}\symbol{"325}}


\usepackage[%authorformat=allreversed,
authorformat=year,titleformat=colonsep,commabeforerest,titleformat=italic,human]{jurabib}

\jbdotafterbibentry

\renewcommand*{\biblnfont}{\textrm} 
\renewcommand*{\bibfnfont}{\textrm} 
\renewcommand*{\bibelnfont}{\textrm} 
\renewcommand*{\bibefnfont}{\textrm}
\renewcommand*{\bibtfont}{\textit} 
\renewcommand*{\bibjtfont}{\textit}
\renewcommand*{\bibbtfont}{\textit} 
\renewcommand*{\bibapifont}{}

\renewcommand*{\bibpldelim}{(} 
\renewcommand*{\bibprdelim}{)} 

\renewcommand*{\bibapifont}[1]{{\quotedblbase}#1``}

\renewcommand*{\bibbtsep}{In: } 

\citeswithoutentry{test}

%======== 本文ここから ======================
\begin{document}

\include{abrr}

\section{Research Aim,Target and Material}
This list registers verb forms which appear in the Pāli Tipiṭaka.
The Pali verb system,or Middle Indic verb system generally,is multifaceted and complex.
It needs the groundwork that opens a path through the jungle of the verb forms.
The aim of my research,therefore,
is to offer a basis of the lingustical investigation of the Pāli verb system as well as of the philological study of the Pāli texts.

As the name of the project says,the literal source is the Pāli Tipiṭaka.
That means,my source is exclusively the canonical Pāli texts.
Verbal forms in non-canonical texts such as commentaries,chronicles,later anthologies and so on are not my target.
Furthermore,paracanonical texts namely Suttasaṅgaha,Nettipa\-ka\-raṇa,
Peṭakopadesa and Milindapañha which have been added to the Khuddhakanikāya in Burma are not material.

Also the prosa of Jātakas is excluded,
because it does not have the status as canonical text for the reason of its later texual fixing.%
\footnote{On the relationship of the Jātaka-verses and prosa; \citet[2-3]{vonhinueber:1998}.}
The only one exception is a part of the prosa of Kuṇāla-ja, which is written in the \textit{Veḍha}, i.e. rhythmical prosa.
This part seems to be old, and the Asian traditions accept the authority of this prosa, therefore regard it as canonical.%
\footnote{Bolle Kunala Ja 169-175 Bechert 1988}.
Thus we take this part exceptionally into our consideration.

The verb forms in the later texts are referred to on occasion,only if they are relevant to our list.
As a sequel to it,this rule applies to `prescriptive verb forms' that are registered only in grammar books.

The list does not include denominatives in the Pāli Tipiṭaka.
They will be collected and listed by Dr. Timothy Lighthiser.

\section{Lemma}
On the grounds that the drastic change of the structure of the verbal systems,
which occurred with the transition from OIA to MIA,the verbal "roots" cannot be considered as origin of all the verbal forms.

According to \citet[78]{Edgerton:1954},all regular Middle Indic verb forms may most conveniently be regarded,descriptively,as based on present stems.
Verb forms which are not so made are to be recognized as irregularities,chiefly relic forms.
The present stems are to be obtained by striking off the 3. sing. present ending.

Following this theory,this list does not list the roots,but the 3. sig. present of the Pāli verbs.

\section{On the analytic conjugation of 1 sg.}
As \citet{Smith:1932} keenly observed, one can recognize the analytic conjugation of 1 sg. in Pāli i.e. \textit{ramahaṃ}, Ja V 112,31* instead of \textit{ramāmi};
\textit{nirajjahaṃ}, Ja VI 502,34* instead of \textit{nirajjāmi}.
a personal pronoun, \textit{ahaṃ}, is adhered, respectively aggultinated, to a stem.
These forms can be regarded as a precursor of 1 sg. \textit{-a\~{u}} in Apabhraṃ\'{s}a.%
\footnote{\citet[9]{pisani:1952} rejects only Smith's misleading argument that the Apabhraṃ\'{s}a form \textit{-a\~{u}} should have influenced upon that of Pali.
Nevertheless, Pisani admits that the analytic form in Pāli continued as \textit{-a\~{u}} in Apabhraṃ\'{s}a.}
Futher examples of the genuine analytic forms are collected by Oberlies: Grammar \S 46 pp. 217-218 (indicative present); 225 (optative.)
In the optative mood, the analytic conjugation \textit{-eyyāhaṃ}, developed further to produce the medium form \textit{-eyyāhe}.%
\footnote{In later vedic texts, the parallel phomenon occured in the future tense.
See AiG I \S 221 with Nachträge zu Bd. I 255,4-9; III \S 224 d) $\alpha$).}

However, this analytic conjugation seems to presuppose prior stages.
This can be the coalescent form, in which the personal pronoun adheres not to a stem, but directly to a verb ending.
Smith himself seems to admit this, giving example from Bv 26,18 *9 \textit{ahaṃ ... vattayām-ahaṃ}.
Superficially it is caused only by the \textit{sandhi} of the ending \textit{-āmi} and \textit{ahaṃ}.
But the word order, in which \textit{ahaṃ} directly follows the finite verb, is noticeable,
and it is also strange that \textit{ahaṃ} is twice used.%
\footnote{See also \citet[114-116]{bloch:1927} = Recueil 142-144.}
This indicates us as follows: Although the understanding of the pronoun still remains,
the coalescent \textit{vattayām-ahaṃ} behaves like a single word,
pretending the conjugation of 1 sg.
%A superficially pleonastic pronoun, connected with a verb,
%can be regarded as a part of conjugation.

%Furthermore the following passage from Mvu is relevant:
%yaṃ nūnāhaṃ kalpānāṃ śatasahasraṃ sthātum icchāmy ahaṃ
%
%Further; von Hinüber: Überblick, gives a good example from Vin and Mvu I
%icchām' ahaṃ bho mahāsamaṇe brahmacariyaṃ carituṃ,

Following Smith, \citet[312-313]{Bechert:1953} collects examples of those coalescent forms of various verb categories in Ap.
They, however, are found also in prosa, especially in dialog.

Smith seemingly enlarged his concept of the analytic conjugation, or at least realized the presupposed process, heading for the analystic conjugation.
Thus, CPD (s.v. \textit{ahaṃ} 2. \textit{Morphology}) gives numerous examples of \textit{ahaṃ}, connected with verbs as coalescence.
And it categorizes the forms which are collected by \citet{Smith:1932} to the stage
$\epsilon$) , in which \textit{ahaṃ} replaces \textit{-āmi} i.e. the genuine analytic form.
The rest of the stages is as follows:
$\alpha$) \textit{ahaṃ} is seemingly pleonatic in a sentence.
$\beta$) \textit{ahaṃ} is put after the finite verb, and seems to be more or less united with the verb.
$\gamma$) \textit{ahaṃ} is used twice in one sentence, as pronoun and as adherence to the verb ending.
$\delta$) the beginning of the analytic optative, in which \textit{ahaṃ} adheres directly to the conjugated optative with the \textit{sandhi}, or with the elision of a vowel.
So, CPD represents that $\epsilon$) stage would be final.
But it is not clear whether the categories from $\alpha$) to $\delta$) represent the chronogical development or not.

Each stage needs more detailed investigation in order to describe the development of the analytic conjugation.
Thus, it can be useful to register and note clearly the coalescent forms of 1 sg.
From this perspective, this list records them as independent, especially after the stage $\beta$),
because the stage $\alpha$) needs a more precise observation of each sentence.

%Apart from the problematic poetical texts, this coalesent forms are found in the prosa.
%Its usage indicates that this form played a roll on the semantic aspekt.
%Whereby it is problematic whether it is right to regard \textit{ramā\textbf{m} 'ahaṃ} as
%intermedial stage to \textit{ramahaṃ}. Why disappears \textit{m}?
%Mv I 72 mokṣayiṣye 'haṃ prajāḥ parimuktaḥ //

\section{On 1 pl. of the present indicative and imperative \textit{-āmase/i}}
An Overview is provided by G § Überblick § Oberlies: Grammar §


Pj II 43,23-28.
carāmase iti carāma, yaṃ hi taṃ sakkatena carāmasī ti vuccati, taṃ idha carāmase iti, aṭṭhakathācariyā pana se iti nipāto ti bhaṇanti,
ten'eva c'ettha āyācanatthaṃ sandhāya carema se iti pi pāthaṃ vikappenti; yaṃ ruccati, taṃ gahetabbaṃ.
whichever one pleases, that is to be taken
See Norman 
As Norman points out that the Sn-commentator aquaints himself with the Sanskrit (sakkata), or better Vedic.
The -āmase ending is connected with ved. -āmasi.

Yet, according to Pj II, the other commentators see differently:
They regard se as indeclinable (nipāta) and alternate (vikappenti) the reading with \textit{carema se} concering the sense of supplication (āyācanatthaṃ sandhāya).
Their view of se as indeclinable favours that āmase ending is difficult to understand already at the time of Sn-commentary.
Actually we find occasionally the explation that \textit{-āmase} should be separated into \textit{-āma} and \textit{se}. 
For examples, Sadd 513,16ff, insist to analyse bhavāmase to bhavāma and se, and gaṇhāma se to gaṇhāma se.
So, Sadd registers the following forms:
dadāmase Sadd 369,30 yamāmase Sadd 412,9 gacchāmase Sadd 463,7 (imper.) āgamayāmase Sadd 473,23 quote
And vadāmase Sadd 386,25 (not used) kubbaamase 511,6 imper. (not used.)

Moreover, Sadd reject also \textit{karomase} and insists to analyse it to karoma and se.
See note on karomase,  $nearrow$ karoti

As Sadd realize \textit{-āmase} ending only for the limited number of the verbs
and it rejects this ending for bhavati and gaṇhati at the same time,
Aggavaṃsa seemingly has difficulty in categorizing the \textit{āmase} ending, showing his inconsistency in dealing with it.

While the ved. \textit{-āmasi} ending is active, \textit{-āmase} is categorized as middle. G
For example, Sadd 369,30, indicates also that \textit{dadāmase} is middle.
Oberlies: Grammar 

See also Norman: EV II note 370-371, on m.c. \textit{ramāmas}\textit{\u{e}}.
Thi 370d; 371b, \textit{ehi ramāmase pupphite vane},
the verse is vaitālīya, so we read ramāmas\u{e} or \textit{si} (Warder: Pāli Metre § 138)

Sadd 842,10
ukkantāmasi bhūtāni pabbatāni vanāni cā ti ettha pi pana vuttirakkhaṇatthaṃ ekārassa ikāro kato ti daṭṭhabbaṃ.

The change of \textit{i/e} is owing to the metrical adjustment:
D II 288,1*, \textit{yaṃ karomase Brahmuṇo samaṃ devehi mārisa}.
Ee Se read \textit{karomase}, Ce Be \textit{karomasi}. Cf. Sv III 740,19 \textit{karomasī ti ... karoma}.
In this case, it is conceivable that the metre could have played a role:
In order to form the cadence of the prior pāda, a short at the fifth syllable is adequate.
\textit{karoma\textbf{si}} indicates more clearly than \textit{-se}, that the fifth syllable is short.
Cf. Sadd IV 1150, 8.1.3,2.
Otherwise, if we allow to adopt \textit{-se}, this must scan u.
According to Überblick § 114, the short \textit{\u{e}}, which the law of the two morae produced, is prononced similarily as the short i.
So the wavering e and i is thereby expected.

dadāmase Sadd 369,30
Khn-Ja.txt:ja3/13115/ yadi hessati dassaama/ asante ki.m dadaamase
Khn-Ja.txt:ja5/31722/ nivesanaani rammaani maya.m bhoto dadaamase
Khn-Ja.txt:ja5/31723/ atha/ vaa/ a.gge vaa magadhe/ vaa/ maya.m bhoto dadaamase
Khn-Ja.txt:ja5/31725/ upa.d.dha.m vaapi rajjassa maya.m bhoto dadaamase

vadāmase Sadd 386,25
kubbaamase 511,6 imper.

yamāmase Sadd 412,9
4Nikaya-MN.txt:mn3/15411/ pare ca na vijaananti/ mayam ettha yamaamase
Khn-Dhp.txt:dhp/00205/ pare ca na vijaananti/ mayam ettha yamaamase
Khn-Ja.txt:ja3/48811/ pare ca na vijaanantimayam ettha yamaamase
Khn-Ja.txt:ja3/48909/ naama/ te/ ettha sa.mghamajjhe kolaahala.m karontaa maya.m yamaamase uparamaama
Khn-Th_Thii.txt:thera/03317/ pare ca na vijaananti mayam ettha yamaamase
Khn-Th_Thii.txt:thera/05213/ pare ca na vijaananti mayam ettha yamaamase

gacchāmase Sadd 463,7 (imper.)
Khn-Ja.txt:ja5/07811/ ohaaya gacchaamase sabbakaame
Khn-Ja.txt:ja5/20021/ ubho va gacchaamase assama.m ta.m
Khn-Ja.txt:ja5/20023/ tattha ubho va gacchaamase ti mama.m/ pitu aarocetvaa ubho va

āgamayāmase Sadd 473,23 quote
Khn-Ja.txt:ja6/08828/ sa.msumbhamaanaa attaana.m kaalam aagamayaamase/ ti

bhavāmase Sadd 513,16ff. (bhavāma se)
Ja VI 567,10
gaṇhāmase Sadd 513,17ff. (gaṇhāma se)
Ja VI 182,13

``akaramhasa te kicca.m; okkantaamasi bhuutaani;
suta.m neta.m abhi.nhaso, tasmaa eva.m vademase''ti
aadiisu;


Khn-Ap.txt:Ap/05901/ athettha satthaa aaga.jchi siisa.m mayha.m paraamasi
Khn-Ap.txt:Ap/18504/ adhivaasesi bhagavaa vattha.m/ hatthena aamasi

Khn-Ja.txt:ja4/04118/ tattheva agaamasi/ pottiyo tassa santike yeva raajaana.m
Khn-Ja.txt:ja6/55501/ okandaamasi/ bhuutaani pabbataani vanaani ca


\section{On 1. pl. of the optative}

One can find \textit{emu}
jaanemu, thera/10403/ daalemu, pucchemu, vijahemu ja3/13112, vindemu sn1/21709, viharemu
ja6/25930 vitaremuu ti vitareyyāma

Oskar quotes Sadd 839,9 in order to support his theory.
Sadd 839,9
kvaci eyyāma-ss' emu
In some places, [the ending] emu [substitues for]`eyyāma'.

Indeed, Sadd teaches the emu-ending of the optative.
However, it is not to be looked over, that this sūtra contains `kvaci':
The substitution of the ending is observed `in some places', but it is possibly not obligated.
So, Aggavaṃsa merely reports on the emu ending in canon, but neither more nor less.

Indeed verbs with ema ending can be mostly regarded as indicative presens or rarely imperative:
mn3/29314/ aakaaraa/ ime anvayaa/ yena maya.m aayasmanto eva.m vadema

Perplexing is the example from ja4/16411, taremuuti tarissaama
Perhaps it is about to the syntactic exegesis, i.e. this opt. form could be interpreted in the future sense.

sn4/38719/ saariputtako.t.thika/ pema
mn1/23427/ mantema
ja4/16325/ ima.j ca diipa.m na pariccajema
ja4/16906/ se.t.thi raajakumaara.m upasa.mkamitvaa/ saami maya.m tumhe posema
ja4/20707/ mante maya.m taadisake na dema
ja6/17317/ karosi/ kima.gga pana maya.m ye paravadhena/ jiivika.m kappema
Khn-Ja.txt:ja6/32108/ yathaa vimaana.m punam aavasema
Khn-Vv.txt:Vv 902  kim ida.m kusala.m aacarema


%Obvious examples according to context are Ja VI 173,17 
%mayam ye paravadhena jīvikaṃ kappema

However, the following examples, though they are extremely rare, go against Oskar.
Sn 447 ap' ettha mudu vindema Pj II 393,11  vindemā ti adhigaccheyyāma (v. l. adhigacchema, adhigaccheyya)
Vv 902  kim ida.m kusala.m aacarema
Vv-a 242,2 ācareyyāma, 
ja5/23430/ appeva naama passema
ja5/23433/ jiivita.m pavattissatiiti diipeti/ passemaati api naama passeyyaama

Sn 898 idha' eva sikkhema, ath' assa suddhiṃ
sn1/20929/ sikkhema susiilyam attano
They can be opt. but does not have a support from cty.

Finally Oskars assertion must be modified:
The form of 1 pl. opt in Pāli canon is that with \textit{emu}-ending,
However, there are rare examples with \textit{-ema} ending.

\paragraph*{On the future forms on -(\textit{i})\textit{hiti}}
Future forms on -(\textit{i})\textit{hiti}, is observed first in the Middle Indic.
We still have not reached a consensus on this new formation of the future, and this is still a matter of controversy among scholars; see \citet{Milizia:2011}.
For the present purpose, we follow the scheme of \textit{-iṣyasi} etc. > \textit{*-i}(\textit{s})\textit{sisi} etc. with saṃprasāṇa > \textit{ihisi} etc.; Überblick § 466-468, \citet[78ff.]{berger:1955}. Cf. \citet[155ff.]{Tedesco:1945}.
For 1 sg., the long vowel in \textit{-iṣy\textbf{ā}mi}, interrupts this development and it remains \textit{-issāmi}.
So, 1 sg. amg. \textit{bhaṇihāmi}, pl. \textit{-āmo} etc. are possibly later and sporadic examples; \textit[43]{schwarzschild:1953}. Cf. Milizia, op. cit., 28-29, though he does not follow this theory.

This phenomenon should have caused a disproportionate paradigma, i. e. 1 sg. \textit{issāmi}, 2 sg. \textit{*ihihi}, 3 sg. \textit{ihiti} etc.
Thereby 2 sg. \textit{*-ihihi} should have been changed into \textit{-ihisi}, modeled after \textit{-issasi}, which preserves \textit{s} between two different vowels. 
Consequently, the -\textit{h}- future should have started from 2., 3 sg. and 3. pl.; Überblick § 468; Schwarzschild, \textit{op. cit.}, 43-44.

The other direction of the adjustment runs to Ś (Aśvaghoṣa) \textit{gamiss\textbf{idi}}, thereby \textit{-issati} is modified to \textit{-issiti}, modeled after \textit{-ihiti}.
Lüders: Brückstücke, 47f.
%
%
%Fut. \textit{kāha/i-} 
On the fut. \textit{-adi/ -idi} in Gāndhārī, see \citet[103-105]{caillat:1977-78}.
%adi has been written when the stem is clearly seen to be that of the future -iṣadi, otherwise idi is used.

Even then, the change of \textit{s} > \textit{h} remains unsettled.
Up to now, the phenomena in the Prākrit, such as loc. sg. m/n. \textit{-ahiṃ}: skt. \textit{-asmiṃ} or gen. sg. \textit{puttaha, puttaho}: \textit{putrasya},
%< puttassa < putrasya, putrasas.
are applicated for the explanation of this change. see \citet[571]{gray:1936}; \citet[fn2, p. 20]{gsjunko:1991}.
% postulates; \textit{kariṣyati} > \textit{*kariṣyiti} > \textit{*karissiti} > \textit{karihiti},
%whereby she refers to the loc. sg. m/n. \textit{-ahiṃ} < \textit{-asmiṃ}
%in Prakrit as example of the quantitative reduction without compensation.
%Much earlier, \citet[571]{gray:1936} explained this phenomenon in a similar way with the genetive singular:
%puttaha, puttaho < puttassa < putrasya, putrasas.
On the other hand, \citet[155-156]{Tedesco:1945} expresses his doubt on the change of  \textit{-ahiṃ} < \textit{-asmiṃ}.
%and possibly not a phonetic change; Überblick § 222, 
%Therefore We should be cautious about it, to explain the change of s > h simply by the %\textit{-ahiṃ} type of the locative singular, or \textit{-aha} of the genetive.
Smith, \textit{op. cit.,} 177 and 181, analysing \textit{karihiti},
is sceptical about such explanation, in which the verb form is analysed by means of the nominal case ending for the first place.
Bloch/Muster, 69, claims also that the nominal endings should be kept away from the verb endings regarding the change of \textit{s} > \textit{h}, and suggests the morphological analogy for its background.
As Überblick \S 221, points out, the condition, in which the change of \texit{s} > \textit{h} occurs, is not explicated yet.
Under these circumstances, it is not easy to reach a consensus over the issue of the origin of the \textit{h} future.

%For example, \citet[\S 8]{pisani:1952} expresses his doubt about its phonetic 
%explanation and tried to handle it as a morphological issue.
%Although \textit{Milizia:2011} does not follow Pisani's theory,
%Milizia assumes the same attitude as Pisani.
%The ideas of both scholars are brilliant, but needs further examination.
%It is perplexing that imperative marker is suddenly used for the future marker.
%We see the close relationship or mutual interference between the opt and imper.,
%and aor. and fut., possibly also the opt and fut. in MIA.
%However, imperative and future are separated from each other.

\citet[§ 7]{pisani:1952} explains that \textit{h} could be taken from the 1 sg. of the periphrastic future, like \textit{hantāhaṃ}; \citet[fn. 159, p. 79]{berger:1955} offers a counterargument.
Pisani's argument starts from that the change of \textit{ss} into \textit{h} of the future stem should not be a phonetic issue but morphological one.

Like Pisani, \citet{Milizia:2011} deals with this issue morphologically.
Thereby he, op. cit., argues that this \textit{h} should be taken from the 2 sg. of the imperative \textit{hi}.
%Though he has certainly a keen wit to point important problems,
%but he tends to interpret the material for the convenience of his theory,
%and to underestimate the counterevidence.
Although he offers an informative argument, we finally do not see any necessity to introduce his theory.
As \citet[300]{Thieme:1981}, future and aorist show occasionally their close interrelationship with each other, both morphologically and semantically.
Ex. 2 sg. conditional = futur imperfect, \textit{agrahaiṣyat}, formed from aor. \textit{agrahaiṣaṃ}.
\textit{aprākṣyaḥ}, formed from aor. \textit{aprākṣaṃ} in the potential function, `...dürftest du ... gefragt haben...'
%See \citet[324]{Alsdorf:1936}, \textit{dacchīhaṃ c'ahaṃ} `and I beheld'.
In this manner, the future and the aorist are interrelated with each other.
And the optative and the future have also a similar close interrelationship.
But Milizia, op. cit., indicates less convincingly the interrelationship between the imperative and the future.
%In this sense, milizias argument is not convincing enough.

%Because the condition, in which the change of \texit{s} > \textit{h} occurs,
%is not explicated yet,
%the question about the origin of the \textit{-h-} is still open
%and thus there is no common understanding among scholars.
%Smith, who explains this form from aorists consisting the vowel \texit{i}.
%\citet[79, fn 161]{berger:1955} rejects it and postulates saṃprasāraṇa.
%Osakar 

See also remarks on \textit{\textbf{karoti}} and \textit{\textbf{harati}}.





\pagebreak

\section{List}
%
%
%1
\begin{center}
{\Large
\textbf{akkhati} `to tell'
}
\end{center}

\begin{description}[leftmargin=\parindent]
\item[ety.]
\textit{ākhyāti}
\end{description}

\begin{description}[leftmargin=\parindent]
\item[pres.] 1 sg. \textit{akkhāmi}, Vin V 97,23*, S I 123,17*([1998] 271,6*), Sn 172.
2 sg. \textit{akkhāsi}, Ja IV 42,7*; VI 507,30*.
3 sg. \textit{akkhāti}, Vin II 202,5*, Sn 87, Ja V 249,8*.
1 pl. \textit{akkhāma}, Ja VI 518,3*.
\item[imper.] 2 sg. \textit{akkhāhi}, Vin V 144,18*, Sn 421, Ja VI 318,20*.
2 pl. \textit{akkhātha}, Ja V 390,18*, VI 577,2*.
\item[opt.]
3 sg. \textit{akkheyya}, Ja IV 226,9*.
\item[fut.]
1 sg. \textit{akkhissaṃ}, Vin V 144,7*. Sn 997.
%Pv 529.
3 sg. \textit{akkhissati}, Pv 579.
\item[cond.]
[3 sg. \textit{akkhissa}, Pv 579] (see Note)
\item[pret.]
1 sg. \textit{akkhiṃ}, Ja V 77,22 = 80,1*.
\textit{akkhāsiṃ}, Ap 612,9.
2 sg. \textit{akkhā}, Ja VI 359,20*.
3 sg. \textit{akkhā}, Ja II 152,12*; IV 271,17* = 394,29*.
\textit{akkhāsi}, Vin I 336,10*, D II 123,9*, Sn 251.
3 pl. \textit{akkhaṃsu}, Ja III 481,8*.
\item[ppp.] \textit{akkhāta-}, Vin II 284,21, D II 118,27, Sn 172. 
\item[ger.] \textit{akkheyya-}, S I 11,22*ff ([1998] 23,16ff) = It 53,24*ff, Sn 808.
% > Nidd I 127,20ff.
\item[inf.] \textit{akkhātuṃ}, Ja V 58,14*, Nd I 127,20.
\item[abs.] \textit{akkhāya}, Sn 829 > Nidd I 169,19ff.
\item[pass.]
\textbf{pres.}
3 sg. \textit{akkhāyati}, M I 24,28.
2 pl. \textit{akkhāyatha}, Vin IV 12,16.
3 pl. \textit{akkhāyanti}, Vin IV 12,6; 7.
\textbf{fut.}
3 pl. \textit{akkhāyissanti}, Vin IV 12,12.

\item[deriv.]
\textit{ākhyāta}(\textit{r}), D I 4,18, M III 8,13, Kv 229,1.
\textit{ākhyāna-} `announcing, preaching, description', M III 167,20, D I 6,12.

\end{description}

\paragraph*{Stock Phrase}
\textit{guyhañ ca tassa akkhāti guyhassa pariguyhati},
A IV 31,21 $\approx$
\textit{guyhañ ca tassa n'akkhāti / akkh- tassa guyhaṃ na/ca gūhati},
Ja IV 197,22; 198,8*.

\textit{iti buddho abhiññāya dhammam akkhāsi bhikkhunaṃ},
D II 123,9*; A II 2,3*; IV 106,5*; Kv 115,8*-9*.

This verse appears in the Mahāparinibbāna-suttanta.
For the A IV 106,5 the verse is somehow used independently of the scene concerning the Mahāparinibbāna.
\textit{bhikkhunaṃ}, in D II 123,9*, is \textit{m.c.} for the \textit{śloka-pathyā},
while the other text adopt \textit{-ūnaṃ}.
See further \citet[320]{Franke:1909}.

\paragraph{Note}
For \textit{akkhāti}, the suppletive verb is \textit{ācikkhati}; CPD ss.vv. \textit{akkhāti}, \textit{ācik\-kha\-ti}.


\textit{akkhāhi}, Ja VI 318,20*-21* = 321,25*-26* \textit{akkhāhi no t' āyaṃ mudhā nu laddho / akkhehi no t' āyaṃ ajesi jūte}.
Read with Be Se in spite of the metrical problem (- - u for \textit{tr̥ṣṭubh}-break).
On the other hand, Ee Ce read \textit{akkhehi} for both passages.
G \S 139.2, based on the reading of Ee, categorized \textit{akkhehi} as an \textit{e}-verb.
CPD ss.vv. 2\textit{akkha, akkhāti} and Oberlies: Grammar \S 45 fn. 5, doubt this view.
Based on the reading of Be, \citet[50-51]{alsdorf:1971} adopts \textit{akkhāhi} for 20*, and \textit{-ehi} for 21* which is inst.pl. of \textit{akkha}.
So far, this is a sound conjecture.
But his further renewing of the text, \textit{akkhāhi: 'nenāsi mudhā nu laddho}, lacks a textual evidence.

\textit{akkhissa}, Pv 579,
\textit{suto ca dhammaṃ sugatiṃ akkhissa}.
If it were heard [by you], he would tell about the \textit{dhamma} of happy existence.
Seemingly it could be a conditional, but it does not fit to the context very much.
Be Ce omit this line, while Ee Se preserve it.
However, it seems that Se had adopted the line from Ee.

\textit{akkhā}, Ja VI 359,20* \textit{yam etam} $\sim$ \textit{udadhiṃ mahantaṃ}
Cf. Ja VI 359,28' \textit{... akkhā ti ... tvaṃ akkhāsi vadesi.}

\textit{akkhaṃsu}, Ja III 481,8* \textit{pubbe va m' etaṃ} $\sim$ \textit{mātā pitā ca bhātaro}.
Cf. Ja III 481,11 \textit{ācikkhiṃsu}.

\textit{akkheyyaṃ}, Sn 808 > Nidd I 127,20ff.
Norman: Sn transl. Note 808, CPD, Cone s.v. \textit{akkheyya}, connect it with skt. \textit{ākhyeya.}
Otherwise, \citet[54]{Jayawickrama:1978} renders the word to \textit{a-kṣayya} < \textit{kṣi}.


\textit{akkhāya}, Sn 829.
Cf. Nidd I 169,21 \textit{vādaṃ akkhāya ācikkhitvā}.




%?really?



%
%
%2
\begin{center}
{\Large
\textbf{agghati} `is worth,deserves'
}
\end{center}

\begin{description}[leftmargin=\parindent]
\item[ety.]
\textit{arhati}
Skt. \textit{arghati}
(EWA  \textit{ARH}
%Verdienen,wert sein
and
\textit{arghá}-%
%Wert,Preis
CDIAL 632
SWTF \textit{argh})
\end{description}

%√%
%\textit{arh}
%"to be worth,deserve".
%see also
%\textit{arghá}-
%"worth,price"\\

\begin{description}[leftmargin=\parindent]
\item[pres.]
3 sg. \textit{agghati},
Vin III 67,36. A III 50,12. Ja III 142,24*.

3 pl. \textit{agghanti},
Ja II 425,8*; IV 66,25*

\item[ger.]
\textit{agghiya-}
Ja VI 516,8*; 517,17*; 533,1*
(\textit{paṭiggahītaṃ yaṃ dinnaṃ sabbassa agghiyaṃ kataṃ})
$\approx$ Ja V 324,5*.
\end{description}

%[agghāma Ja VI 174,19 mayaṃ tassa paricārake pi na agghāmā]
%3. sing. ta.m agghaapenta.m na pa.jca maasake -ati Vin III,67,36.
%kim -ati ta.n.dulanaa.likaa ca Ja I 126,5 (see Corrections and Additions to vol. 1.)
%candanaphalaka.m -ati adhikasatasahassa.m AN III 50,12

\begin{description}[leftmargin=\parindent]
\item[caus.] `values' \textbf{pret.} 3 sg. \textit{agghāpesi}, Vin III 67,35-36.
\textbf{part. pres.} \textit{agghāpentaṃ} Vin III 67,36.
\textbf{abs.} \textit{agghāpetvāna} Ap 63,5 / \textit{agghāpetvā} Ap 106,21.
\end{description}

\paragraph*{Stock phrases}
\textit{kalaṃ n' agghati soḷasiṃ},
Dhp 70. Th 1171. Ap I 133,7.
Prosa: SN V 343,22-23. A I 166,4-5; 168,6-77; 213,5; II 70,11; IV 252,9; 256,21; 261,2; V 235,34; 251,23
\textit{- n' agghanti -}.
in prosa: A III 365,11.
Ud 11,23.
It 19-20,25ff.
In verse: Vv 190-193; 720-723.

\paragraph*{Note}
\textit{agghati}: Ja I 126,5* is registered in Corrections and Additions to vol. 1.
Dhp *70d \textit{n' agghati} is authentic,
though Ce Be and Se read \textit{agghati} without ``\textit{n}."
See Ee(1995) note on *70d,Dhp-a I 63,5ff and
\citet[163]{rau:1959}.
\textit{agghanti}: Ja IV 66,9*ff. \textit{kalam pi naagghanti} aginst the law of morae.

 [\textit{agghato}]: part. pres. Thī 386; 394.
Norman: EV II,note on 386 offers a alternative reading \textit{agghato kato} to \textit{aggato kato} (Ee Se Ce \textit{aggato kato},Be \textit{aggito kato}.)
As he cites,the Ce of Thī-a holds a variant reading \textit{agghato hato} but the Be of Thī-a \textit{aggito kato}.
In response to both different lemmata,the wording of the cty differs between the Ce and the Be (See EV II,note on 386.)
Pruitt follows the Burmese reading (see Thī-a 238,24-25 with fn. 7 and 8 as well as Pruitt's translation 323 with fn. 4)
In the context of Thī 386 \textit{aggha-} is not unsuitable but supported only by the Ce of the cty.
\textit{agghato kato} remains therefore a tentative reading.
Although Norman prefers \textit{agghato} to \textit{aggato} in Thī 394 (Ee Be Ce Se and cty \textit{aggato kataṃ},)
he finally chooses \textit{aggato kataṃ} for his translation; Norman: EV II 394 with note.

\textit{agghiya-}: On the \textit{svarabhakti} `\textit{i}' of \textit{agghiya}-; Oberlies: Pāli Grammar §7.13a.
On \textit{agghiya-} which referes to `a column of honour'; see CPD \textit{agghiya-} and Cone \textit{agghiya-}3.


%
%
%3
%\begin{center}
%{\Large
%\textbf{aṅkati,-eti,-ayati} `mark,pierce'
%}
%\end{center}
%\begin{description}[leftmargin=\parindent]
%\item[ety.] \textit{aṅkate},\textit{aṅkayati}
%\end{description}
%\begin{description}[leftmargin=\parindent]
%\item[ppp.] \textit{aṅkita-} Ja II 185,10* \textit{aṅkitakaṇṇako}.
%\end{description}
%\paragraph*{Note}
%Though \textit{abs. aṅketvā} is found in the Ja prosa (Ja I 451,25; II 399,4),
%but the verb itself is seldom used.


%
%
%3
\begin{center}
{\Large
$\textbf{acchati}_1$ `is,sits,remains,continues (doing something)'
}
\end{center}

\begin{description}[leftmargin=\parindent]
\item[ety.]
\textit{ākṣeti}
EWA  $\textit{KṢAY}^2$
%wohnen,verweilen
CDIAL 1031,Turner: Collected Papers 340-356 discusses in detail
BHSD \textit{acchati}
CPD follows G §135 √\textit{ās}
\end{description}
 
\begin{description}[leftmargin=\parindent]
\item[pres] 3 sg. \textit{acchati},  D I 101,23. Ap 269,16 (\textit{acchatī}). 2 sg. \textit{acchasi}, D I 102,33.
3 pl. \textit{acchanti}, Vin I 160,5. D III 94,19; 94,22 (Ee \textit{acchenti})
\textbf{med.} 3 pl. \textit{acchare} S I 212,28. Ja VI 557,16*; 18*.\\

%[1 pl. \textit{acchāma}]

\item[imper.]
3 sg. \textit{acchatu} Ja VI 506,13*.
2 sg. \textit{accha} Ja VI 522,17*.
\textbf{med} 3 sg. \textit{acchataṃ}, Ja VI 506,8*.
2 sg. \textit{acchassu}, Ja VI 18,27*.
%; 423,7*; 516,16 (all in the same phrase \textit{idh' eva tāva acchassu yāva} ... )

\item[opt.]
3 sg. \textit{acche},
A II 15,6*; It 120,10*.

\item[fut.]
3 pl. \textit{acchissanti}, Vin II 76,3.

\item[pret.]
%is-aorist
3 sg. \textit{acchi}, Vin III 35,4.
1 sg. \textit{acchi}[\textit{ṃ},] Ja VI 17,8* (acchāhaṃ; see cty Ja VI 17,24 acchāhan ti acchiṃ ahaṃ, avasin ti attho.)
\end{description}

%1 sg. \textit{acchāmi} Ja VI 419,36

\paragraph*{Prefix}\mbox{}\\
\begin{description}[leftmargin=\parindent]
\item[sam-] ``to sit down togehter''
	\begin{description}[leftmargin=\parindent]
	\item[pres] \textbf{med.} 3 pl.  \textit{samacchare}, Ja II 67,19*.
	\end{description}
\end{description}

\paragraph*{Stock phrase}
\textit{tatth' acchasi devi mahānubhāve} Vv 94; 101; 303.
\textit{tatth' acchasi pivasi khādasī ca} Vv 906; 1054; 1120; 1140; 1276.

\paragraph*{Note}
\textit{acchasi}: CPD claims that \textit{acchasi} Ja VI 518,6 is 2 sg. fut. from \textit{*ātsyase}.
But \citet[39]{alsdorf:57} adopts the Burmese variant reading (=Bd) \textit{vacchasi} < \textit{vatsyasi}.
And \citet[99]{cone:77} accept his proposal.
The Ee Ce and Be of Ja cty (Ja VI 519,1) read indeed \textit{acchasī ti vasissasi} (Bd \textit{vacchasī ti vasissasi}.)
However,Vess-dīp (written in Chiang Mai,1517) also cites this passage of Ja,but its wording is \textit{vacchasī ti vasissasi}.
So, the old nothern Thai tradition holds \textit{vacchasi}.
%2 pl. \textit{acchatha} ja4/30612/ sama.nadhamma.m karontaa acchatha

\textit{acchenti}: D III 94,19 Ee \textit{acchenti} with v. l. Bmr \textit{acchanti} K \textit{āgacchanti},Be Ce \textit{acchanti},Se \textit{āgacchanti};
94,22 Ee \textit{acchenti} with v. l. K \textit{gacchanti}, Be Ce \textit{acchanti},Se \textit{gacchanti}.
But,Sv III 870,31 (also Se) \textit{acchantī ti vasanti,acchentī ti pi pāṭho. es' ev' attho}.
Cf. Sv-ṭ III 61,1-2 \textit{acchantī ti āsanti,upavisantī ti attho. tenāha vasantī ti. accentī} (Be \textit{acchentī}) \textit{ti kālaṃ khepenti.}
It seems that the variant readings arouse in the relatively old time.
I accept the reading \textit{acchenti} because of its phonologically possible development from \textit{ākṣeti}.
On the general rule of the phonological change \textit{kṣ} > \textit{cch}; Überblick §232-235.

\textit{acchataṃ}: Ja VI 506,8*. Cty Ja VI 506,9-10 \textit{acchatan ti acchatu. idh' eva hotū ti vadati.}

%
%
%4
\begin{center}
{\Large
$\textbf{*acchati }_2$ `mark'
}
\end{center}
\begin{description}[leftmargin=\parindent]
\item[Skt.]
\textit{akṣṇoti}
EWA \textit{AKṢ}.
See also \citet[85]{narten:1964}.
$\nearrow$ \textit{lacchati}; \textit{luñcati}
\end{description}

\paragraph*{Prefix}\mbox{}\\
\begin{description}[leftmargin=\parindent]
\item[nir-]`to castrate'
\item[pret.] 3. sg. \textit{nilacchesi} Thī 437 (eds. nilla-.)
\item[ppp.] \textit{nilacchito} Thī 439; 440 (eds. nilla-.)
\textit{nilicchito} Ja VI 238,18* (so Ee Ce,niluñcito Be Se) cty Ja VI 239,11 \textit{nilicchako uddhaṭabījo} (so Ee Ce,\textit{niluñcito uddhatabījo} Be Se.)
\end{description}

\paragraph*{Note} Thī 437; 440 are to read \textit{nilacch-} respectively (Alsdorf: App. II 243; Norman: EV II §65b.)
Trecnkner assigns \textit{nilacchesi} and \textit{nilacchito} (with single \textit{l}) to √\textit{akṣ}
(Trenckner: PM 55.)
This has the meaning of `to mark,pierce' and refers to pierce a mark into ears of cattle.
Pāli dictionaries assign them rather to \textit{lāñch} (PED,Mizuno and Kumoi.)
Cone adds further possible skt. correspondences; \textit{lakṣayati} and \textit{luñcati} to it.
As verb,they are hardly used with \textit{niṣ} and we find only adj. \textit{nirlakṣaṇa} `featureless' and noun \textit{nirlāñchana}
which is used in Hemacandra's Yogaśāstra III 99 and 110.
This \textit{nirlāñchana} does not designates directly the castration but prohibited actions against cattle.
The excision of testicles (\textit{muṣkachedanaṃ}) is included in the category of \textit{nirlāñchana}.

In the Jaina literature we can further find a relevant expression.
\textit{n/ṇelaccha} `eunuch':
Dhanapāla's Paialacchināmamālā 235 registers \textit{nelaccha} as a synonym of \textit{paṃḍaa} skt. \textit{paṇḍaka}.
Hemacandra's Deśīnāmamālā IV 44 \textit{ṇelaccha}; cty: \textit{ṇelaccho ṣaṇḍho. vṛṣabha ity anye}.
But his Śabdānuśāsana (Hc II 174) is somehow perplexing: \textit{paṇḍitaḥ ṇelaccho}.
In this case we may read \textit{paṇḍaka} instead of \textit{paṇḍita}.

On the ground of these Jaina glossaries and grammar book
Pischel §61 renders \textit{ṇelaccha} to skt. \textit{nirlakṣa} via \textit{*ṇellaccha,*ṇillaccha}.
Pischel simply follows Bühler's opinion; \citet[144]{Buhler:1878} renders it to skt. \textit{nairlakṣa} or \textit{nirlakṣa}.
But,Dośī,the editor of an indian edition of the Paialacchināmamālā,does not render it to skt. in his glossary,
otherwise he always renders the Prākṛt words to skt. He must have seen the difficulty of rendering.

These words express more or less something to do with the castration.
But they scarecely contribute to solve the etymological problem.

The idea of Trencker,thus,can be to the point.
\citet[85]{narten:1964} states clearly that nir-√\textit{akṣ} means `to castrate',that \citet[40]{Delbruck:1896} suggested in the remote past.
We find also \textit{nir-akṣṇoti} ŚB IV 4,2,13,\textit{nír akṣṇuhi} AV 12,4,6 \textit{nír-aṣṭa-} RV I,33,6 and so on in the vedic literature:
Examples collected by Narten: Kleine Schriften 308-312.
Therefore,it would be better to set the point of departure on nir-√\textit{akṣ}.

Can we suppose \textit{*nilacchadi} from \textit{nir}-√\textit{akṣ} with the change \textit{nir}- > \textit{nil}- in the `eastern dialect'?
This could encounter another problem: The change \textit{cch} < \textit{kṣ} is `western' in principle (Überblick §234.)
As Oberlies: Pāli Grammar §18.1 fn. 8,we may assume that there must be a blend or confusion of *\textit{nir}-√\textit{lakṣ(a)y}, *\textit{nir}-√\textit{lāñch} and \textit{nir}-√\textit{akṣ}.
Cf. Berger: Zwei Probleme 77-78.

\textit{nilicchito} or \textit{niluñcito} Ja VI 238,18* is also to analyze as a blend or confusion of *\textit{nir}-√\textit{lāñch} or *\textit{nir}-√\textit{luñch} and nir-√\textit{akṣ}.

On the other hand,\textit{nilañchakā} Ja IV 364,2* (Ee \textit{tilañchakā} Ce Be \textit{nilañchakā} Se \textit{nilañcakā}) seems to have nothing to do with nir-√\textit{akṣ}.
It should be connected with *\textit{nir}-√\textit{lāñch} and be compared with \textit{nirlāñchana} in Hemacandra's Yogaśāstra III 99 and 110.
Cf. Cone (s. v. \textit{nilañchaka}); Toevoegselen II 175.
%In other words \textit{nilañchakā} was taken in Pali much later than \textit{nillacchesi} or \textit{nillacchito}.

%
%
%
%5
\begin{center}
{\Large
\textbf{ajati } `drives,brings forth,hurls'
}
\end{center}

\begin{description}[leftmargin=\parindent]
\item[ety.] \textit{ajati} (EWAi \textit{AJ}.)
Simplex is registered only in later commentaries and grammar books (Pālim-nṭ,Abh-ṭ Dhātup,Dhātum,Sadd.)
\end{description}

\paragraph*{Prefix}\mbox{}\\
\begin{description}[leftmargin=\parindent]
\item[ud-ā-] `to scare (away)' \textbf{sam-ud-ā-} `to scare (away) well'
	\begin{description}[leftmargin=\parindent]
	\item[ppp.] †\textit{udājitaṃ}, †\textit{samudājitaṃ},  S IV 196,23. Paṭis I 162,25.
	See \citet{vonhinuber:1979} = Kleine Schriften 616-619. Cf. CPD (s. v. †\textit{uducita}).
	\end{description}

	\begin{description}[leftmargin=\parindent]
	\item[ppp.] \textit{samudājitaṃ}
	%recorded only as one of alternative readings of the lemma (\textit{udutaṃ cittaṃ samuducittaṃ}) in the Paṭis-a II 469,15-16.
	See above \textbf{ud-ā-}.
	\end{description} 
\item[nir-] `to drive away' 
	\begin{description}[leftmargin=\parindent]
	\item[Pass.] \textbf{pres.} 3 sg. \textit{nirajjati} \textbf{pret.} 1 sg. \textit{nirajji(ṃ)}, Ja VI 505,17* = 505, 32*
	%cty: Ja VI 505,25 \textit{nirajjahan ti nikkhanto ahaṃ}
	Stock phrase: \textit{samhā raṭṭhā nirajjati}, Ja VI 502,34*; 503,4*; 8*; 16*; 23*.
	\end{description}
\item[pa-] `to drive [an animal]'
	\begin{description}[leftmargin=\parindent]
	\item[Caus.] \textbf{pres.} 3 sg. \textit{pāceti} (not in the causative meaning) Dhp 135b. 3 pl. \textit{pācenti} Dhp 135d.
	\textbf{pret.} 1 pl. †\textit{pācayimha} Ap II 593,25 \textit{abhabbaṭṭhāne vajjetvā pācayimha anāvaraṃ} (Ce Be Se \textit{vārayimha anācaraṃ.})
	\textbf{part. pres.} fem.†\textit{pācayantī} Ap II 586,15 \textit{abhabbaṭṭhāne vajjetvā pācayantī anāvaraṃ} (Ce Be Se \textit{vārayitvā anācaraṃ}.)
\item[pari-pa-]
	\item[Caus.] (not in the causative meaning) \textbf{part.pres.}†\textit{paripācento} Ap II 552,23
	\textit{abhabbaṭṭhāne vajjetvā paripācento anāvaraṃ} (Ce \textit{parivajjentī anācāraṃ} Be \textit{vārayantī anācaraṃ} Se \textit{paripācento anācāraṃ}.) 
	\end{description}

\begin{description}[leftmargin=\parindent]
\item[derivative]
\item[parā-aj] \textit{pārājika}:
As we recognize that the vedic verb \textit{aj} is still used in Pāli texts,we may pave the way for the interpretation of \textit{pārājika},
as Smith (Sadd Index s. v. \textit{pārāji}) and in the remote past Burnouf: Introduction à l'histoire du Bouddhisme indien. Paris 1876,2ed. 286 indicated.
See also \citet[62 with fn.14]{vonhinueber:1985}.
Other interpretations: \citet[341-342]{roth:1968},
Levi: Langue précanonique 505-506.

\end{description}
%Uv_1.17ab: yathā.daṇḍena.gopālo.gāḥ.prāpayati.gocaram /
%Uv_1.17cd: evam.rogair.jarā.mṛtyuḥ.āyuḥ.prāpayate.nṛṇām //
%meanig: throw Ja Prosa
%ja6/28115/ maa.nava paasake paajehiiti/ mahaaraaja/ pa.thama.m mama vaaro
%ja6/28116/ na paapu.naati/ tumhe paajethaati/ raajaa/
\end{description}

\paragraph*{Note}
†\textit{udājitaṃ}, †\textit{samudājitaṃ}: Paṭis I 162,25. S IV 196,23
recorded as one of alternative readings of the \textit{pratīka}
(\textit{udutaṃ cittaṃ samuducittaṃ}) only in the Paṭis-a II 469,15-16.
\citet{vonhinuber:1979} = Kleine Schriften 616-619 attests the authenticity of \textit{udājitaṃ}.
Cone (s. v. \textit{udājita}) cites variant readings of Paṭis, Paṭis-a, S and Spk from South and Southeast Asian editions.
	
\textit{nirajji}:  For Thī 93 \textit{sāmaññatthaṃ nirajji' haṃ} there are two readings: \textit{nirajji' haṃ} (Ee Se) and \textit{na bujjhi' haṃ}(Ce Be).
Thī-a reads obviously \textit{na bujjhi' haṃ} because of its paraphrase \textit{na bujjhiṃ,na jāniṃ ahaṃ} (Thī-a 88,16.)
Cone (s. v. \textit{nirajjati}),therefore,is doubtful about \textit{na nirajji' haṃ}.
But Norman: EV II note on 93 does not entirely exclude \textit{nirajji' haṃ}.
The reading could be acceptable if we could also accept \textit{sāmaññatthaṃ} as abl. sg.,because \textit{nirajjati} are used with an ablative.
On the repeatedly discussed but still not settled problem on abl. sg. m.,n. in \textit{-aṃ}; Überblick §304-305.
\textit{nirajji(ṃ)}: Ja VI 505,25 \textit{nirajjahan ti nikkhanto ahaṃ.} See also\citet[169]{Smith:1932}.

\textit{pāceti},\textit{pācenti}: Dhp 135 Ee(1995) suggests that \textit{pājeti} in the Be and Thai manuscripts (H,L) is probably the older reading.
The form \textit{pāceti},however,can be attributed to the `eastern dialect' \textit{pāye(t)i} (Lüders) or \textit{pāyedi} (von Hinüber.)
See also Lüders: Beobachtungen §140 and Überblick §177.
In the Ja prosa,\textit{pājeti} is ocasionally used; examples collected in Lüders: Beobachtungen §140 and Cone (s. v. \textit{pājeti}.)
On the change \textit{c:j} in MIA; Überblick §174-177,Norman: Collected Papers I 96-97; 176-177.

†\textit{pācayimha},
†\textit{pācayantī},
†\textit{paripācento}: They are collected from the Therī-Ap.
The readings at issue vary among printed editions,
and the selection of the readings has substantial influence upon the
interpretation of the Ap verses.
The readings of Ee are nearly isolated and confronted by those of other editions.
But those lines from Ap at issue are very similar,thus we can summarize the problem:
Which of two i.e. \textit{pācayati} or \textit{vārayati} is originally meant?
We take Ap II 593,25 as example.
The Ce,Be and Se chooses \textit{vārayimha anācaraṃ} `we prevented misconduct'.
This is intelligible but may be a \textit{lectio facilior}.
On the other hand,
the reading of Ee \textit{pācayimha anāvaraṃ} `we drove away the matchless' is not clear,because we do not know what is actually the `matchless.'
\textit{anāvaraṃ} `matchless',however,is used also in It 76,7*-8* \textit{jetvāna maccuno senaṃ,vimokkhena anāvaraṃ} `having defeated the matchless army of the death through the emancipation.' 
If the above mentioned Ap-verses refer to the It 76,7*-8*,then Ap II 593,25 can be translated in the following manner:
`having abstrained from the impossible [acts for an arhat],we drove away the matchless [army of the death.]'
Nevertheless,the reading of the passage remains unsettled. We needs further philological treatment.
 Cf. \citet[314]{Bechert:1958} refers to \textit{pācayimha} Ap II 593,25 but does not mention the variant readings in the Sinhalese or Southeast Asian editions.

%
%
%6
%\begin{center}
%{\Large
%\textbf{añcati} `bends'
%}
%\end{center}
%\begin{description}[leftmargin=\parindent]
%\item[ety.] \textit{añcati} EWA $\textit{AÑJ}_1$
%\end{description}
%\begin{description}[leftmargin=\parindent]
%\item[pres.] 1 sg. \textit{añcāmi} Th 750 (possibly wrong reading,cf. Ce Be Se \textit{añchāmi})
%\end{description}
%\paragraph*{Note} Norman: EV I note on 75 chooses \textit{añchāmi} by reason of the cty:
%Th-a II 26,26 \textit{añchāmi}; 28,25 \textit{añchāmī ti ākaḍḍhāmi}.
%In the Ce,Be and Se of the Th-a there is no variant reading.
%Cone (s. v. $\textit{añcati}^1$) also suggest that the reading of Ee is probably wrong.

%
%
%7
%\begin{center}
%{\Large
%\textbf{añcati} `draws (water)'
%}
%\end{center}
%\begin{description}[leftmargin=\parindent]
%\item[ety.] \textit{añcati} EWA $\textit{AÑJ}_2$
%\end{description}
%
%
%\begin{description}[leftmargin=\parindent]
%\item[derivatives]
%\item[ud-] `to draw out (water)'
%\textit{udañcanī} `bucket for drawing water out of a well' Ja I 417,10. Cf. Skt. \textit{udañcana} 
%\end{description}

%
%
%6
\begin{center}
{\Large
\textbf{añchati} `stretches,pulls'
}
\end{center}
\
\begin{description}[leftmargin=\parindent]
\item[ety.] \textit{āñchati}
CDIAL 1099
\end{description}

\begin{description}[leftmargin=\parindent]
\item[pres.] 1 sg \textit{añchāmi}, D II 291,17.
Th 750 (Ee \textit{añcāmi}.)
\item[par. pres.] \textit{añchanto}, D II 291,16.
\end{description}

\paragraph*{Prefix}\mbox{}\\
\begin{description}[leftmargin=\parindent] 
\item[sam-] `to stretch well'
\item[pres.] 2 sg. \textit{samañchasi}, Vin IV 171,5  \textit{nisīdanaṃ samantato samañchasi} (Ee samañcasi Ce samañjasi Be Se samañchasi)
\item[part.pres.] \textit{samañchamāno}, Vin IV 171,2-3 \textit{nisīdanaṃ paññapetvā samantato samañchamāno} (Ee samañcamāno Ce samañjamāno Be Se samañchamāno)
\end{description}

\paragraph*{Note}
\textit{añchāmi}: Th 750 to read \textit{añchāmi}.
Except for the Ee, Ce Be and Se read \textit{añchāmi}.
Norman: EV I note on 750 chooses \textit{añchāmi} by reason of the cty:
%Th-a II 26,26 \textit{añchāmi}; 28,25 \textit{añchāmī ti ākaḍḍhāmi}.
%In the Ce,Be and Se of the Th-a there is no variant reading.
%Cone (s. v. $\textit{añcati}^1$) also suggest that the reading of Ee is probably wrong.
Trenckner wants to assign \textit{samañchasi} and \textit{samañchamāno} in Vin IV 171 to √\textit{añc} `to bend'.
However the reading of the Vin IV 171,as above,is not fixed.
Horner's translation (95 with fn. 3) choses \textit{sam-añch} because of the cty,Sp IV,884,13-15:
\textit{yathā hi cammakāro cammaṃ vitthataṃ karissāmī ti ito c' ito ca samañchati,kaḍḍhati,evaṃ so pi taṃ nisīdanaṃ}
``as the leather-worker says,`I will make this hide wide,' and pulls it out,tugs it out from here and there,so he (does) to that piece of cloth to sit upon.''
The commentator reads obviously \textit{sam-añch}.
We also accept rather \textit{sam-añch}.

%
%
%7
\begin{center}
{\Large
$\textbf{añjati}^1$ `anoints'
}
\end{center}
\
\begin{description}[leftmargin=\parindent]
\item[Skt.] \textit{anakti}
CDIAL 169 EWA \textit{AÑJ}
\end{description}

\begin{description}[leftmargin=\parindent]
\item[pres.] 3 pl \textit{añjanti} Vin I,203,33.
\item[opt.] 2 sg \textit{añjeyyāsi} S II,281,12.
\item[abs.] \textit{añjetvā},\textit{añjitvā} S II 281,4 (Ee Be \textit{añje-}; Ce Se\textit{añji-})
\end{description}

\paragraph*{Prefix}
\begin{description}[leftmargin=\parindent]
\item[abhi-] `to oil,to grease'
	\begin{description}[leftmargin=\parindent]
	\item[opt] 3 sg.  \textit{abbhañjeyya} S IV 177,2 = Nd I 368,25.
	\item[pret.] 3 pl. \textit{abbhañjiṃsu} Vin III 83,15.
	\item[abs.] \textit{abbhañjitvā} D II 324,17-18.
	\item[part.pres.] \textit{abbhañjanaṃ} Vin I 205,13; 251,30*. Ap 236,5-6; 456,7-8.
	-\textit{nassa} Ap 456,9. -\textit{ne} Ap 315,12.
		\begin{description}[leftmargin=\parindent]
		\item[comp.] \textit{atikkhitta}- Vin I 251,34*. \textit{pāda}- Vin I 205,29; 251,36*. \textit{akkha}- Nd I 484,6.
		\end{description}	
	\item[ppp.] †\textit{abbhatta}- Ja VI 252,20 \textit{kucchisaññamanabbhatto} (Ee Be Ce Se -\textit{abbhanto}. See CPD c. v. \textit{abbhatta}.)
	%\item[derivative] \textit{Abbhañjana},-\textit{dāyaka}: Name of a \textit{Thera} Ap 236,1; 236,11-12; 238,20; 456,13; 456,14; 463,13
	\end{description}
\item[upa-] `to smear'
	\begin{description}[leftmargin=\parindent]
	\item[ppp.] \textit{upatta}- M I 343,35 \textit{bhūmiyā haritupattāya} (Ce Ee) \textit{haritupalittāya} (Be Se)
	\end{description}
\end{description}
\paragraph*{Note}
We find causative forms (\textit{abbhañjayiṃsu} Ja V 376,32 \textit{abbhañjetvā} Ja I 438,20 \textit{abbhañjāpetvā} Ja III 372,25) in the Jataka prosa.

\textit{abbhatta}-: Ja VI 252,20 \textit{kucchisaññamanabbhatto} (Ee Be Ce Se -\textit{abbhanto}.)
Against editions CPD deems  -\textit{abbhanto} wrong and corrects it to -\textit{abbhatto}.
This judgment is pertinent because \textit{abbhanta} is not registered even in Pāli dictionaries from Southeast Asia.
As CPD does,\textit{abbhanto} in the cty Ja VI 253,11-12 should be corrected:
\textit{kucchisaṃñamasaṃkhātena mitabhojanamayena telena abbhatto}
`being lubricated with sesame oil, called restraining of the belly, consisting of the moderate nourishment.'

Cone (s. v. `\textit{abbhanta}') also considers it as wrong reading.
But her comment on Ja VI 252,20* (s. v. \textit{kucchisaṃyamanabbhanta}) differs from that of CPD.
She splits \textit{kucchisaññamanabbhatto} into \textit{kucchi} + \textit{saññama} + \textit{nābhi} + \textit{anta} and interprets it as `whose nave is restraint as regards the belly.'
It is because that she takes an additional passage: \textit{abbhañjitabbo nābhi hotū ti pi pāṭho} to the Ja cty (Ja VI 253 11-12),
which only Be shows us,
into consideration.
Further she corrects \textit{abbhatto} in the cty Ja VI 253,12 to \textit{abbhañjito} because of the Ce reading.
The possibly new ppp. \textit{abbhañjita} is used in Spk I 87,23 (\textit{an-abbhañjite}),Vism 362,26 = Vibh-a 68,18 (\textit{tel' abbhañjite}.)
Therefore it may be possible that the original wording of Ja VI 253,12 could be \textit{abbhañjito}.

At the same time Cone gives another interpretation following the CPD:` lubricated with restraint as regards the belly'.
Because of the collocation with \textit{tela} `sesame oil',\textit{abbhatto} `smeared' is possible.

We ,however, can read differently than Cone and CPD,
consulting with that Burmese additional passage:
\textit{kucchisaññama-nābh' atto} `nave greased with restraint as regards the belly.'


%
%
%8
\begin{center}
{\Large
$†\textbf{añjati}^2$ `goes'
}
\end{center}

Doubtful

\begin{description}[leftmargin=\parindent]
\item[imper.] \textit{añja} 2 sg. Vin IV 5,20; 33 ($\approx$ Ja I 192,5 \textit{añja kūṭa}; 29 \textit{añja bhadra}.)
\end{description}

\paragraph*{Note}
\textit{añja}: Vin IV 5,20; 33 ($\approx$ Ja I 192,5; 29 in prosa.)
The reading of the passage is not settled:
Vin IV 5,20; 33 Ee \textit{gaccha kūṭa}; \textit{gaccha bhadra} Be \textit{gaccha}; \textit{accha} Se \textit{añcha}; \textit{añcha}
Ce \textit{añja}; \textit{añja.} 
Ja I 192,5; 29 Ee \textit{añja kūṭa}; \textit{añja bhadra} Be \textit{gaccha}; \textit{gaccha} Ce Se \textit{añcha}; \textit{añcha} 

Cone (s. v. $\textit{añjati}^1$),because of the Dhatup 69 and Dhatum 74,
admits that \textit{añjati} has the meaning of `goes' and interprets \textit{añja} as imper.
And she claims that the reading of the Vin should be changed to \textit{añja}.

At the first place Trenckner interprets \textit{añja} as imper. of √\textit{aj}.
On the other hand,Kern: Toevoegselen I 76 regards it as imper. of √\textit{añj} and translates `ga! trek op!' =`go! pull on!'
Kern's idea is followed by PED which admits the origin of the word as imper. of \textit{añjati}.
But PED says further that \textit{añja} loses the meaning of imper. and is already employed as an adverb.
CPD,following the idea of PED,wants to explain that \textit{añja} as indeclinable and gives it the meaning `go on! pull on!'

But according to EDS,skt. \textit{anakti} or \textit{añjati} does not mean `goes' but `anoints.'
Skt. \textit{añcati} could mean `goes.'

It seems that we depart better rather from skt. \textit{añjaḥ} `quickly' as BHSD (s. v. [añja.) Cf. KEWA $\textit{añjaḥ}_2$.
Or we could otherwise emend \textit{añja} to \textit{añcha} `pull' on the ground for Se.
Thus,it is still a open question whether the verb {*añjati} could mean `goes' in Pāli.

The mentioned passage of Ja is quoted from the Nandivilāsa-ja (Ja I 191-193.)
This Ja is included in the Pācittiya section of the Vin i.e. Vin IV 5-6.
According to \citet[188-189]{vonhinueber:1998},both version differs from each other in the structure of the story and the wording,
and the version in Vin is considered as older than that in the present Ja collection.

%
%
%11
%\begin{center}
%{\Large
%$*\textbf{aṭati}$ `roams'
%}
%\end{center}
%Simplex is used in the later texts (Cone s. v. \textit{aṭati})
%
%\begin{description}[leftmargin=\parindent]
%\item[skt.] \textit{aṭati} EWA \textit{AT}
%\end{description}
%
%\begin{description}[leftmargin=\parindent]
%\item[derivatives]
%\item[comp.] \textit{aṭanaka-gāvī} `wild cows' Ja V 105,25*. 
%\end{description}

%
%
%9
\begin{center}
{\Large
$*\textbf{aṭṭati}$ `hurts, afflicts'
}
\end{center}
\begin{description}[leftmargin=\parindent]
\item[skt.] \textit{ardati}
\end{description}

\begin{description}[leftmargin=\parindent]
\item[ppp.]
\textit{aṭṭa} Ja V 53,20*. Sn 694.
\textit{aṭṭita},
Ja VI 524,16*.
\textit{addita}, Thī 328.
\end{description}


%
%
%10
\begin{center}
{\Large
$*\textbf{aṇati}$ `speaks,recites'
}
\end{center}

\begin{description}[leftmargin=\parindent]
\item[skt.] \textit{aṇati} PW \textit{aṇ} `tönen'
\end{description}

\paragraph*{Note}
used only in the etymology of \textit{brāhmaṇa} in the commentaries:
\textit{brahmaṃ aṇatī ti brāhmaṇo} Sp I 111,12. Ps I 109,22. Bv-a 67,33. Ud-a 58,13.\\


%
%
%11
\begin{center}
{\Large
\textbf{atthi} `is, is found'
}
\end{center}
\begin{description}[leftmargin=\parindent]
\item[skt.] \textit{asti} CDIAL 977
\end{description}

\begin{description}[leftmargin=\parindent]
\item[pres.]
1 sg. \textit{asmi},
Vin I 32,19, A I 284,28, Ja II 169, 7* (\textit{'smi}).
%Ja VI 572,16*.
%\textit{smi},
%Ja II 169, 7*.
\textit{amhi},
S V 356,10 (\textit{'mhi}), Th 1211. Thī 223.
%(\textit{ahaṃ amhi kantasallā ohitabhārā} Appendix II 239.)
2 sg. \textit{asi},
Vin I 87,23 (\textit{'si}), Sn 426*, Ja V 491, 25*.
%S I 8,3([1998]17,8*),
%\textit{si},
%Vin I 87,23. M I 170, 36.
%Vv 1218.
%\textit{mhi},
%S V 356,10. Thī 11.
3 sg. \textit{atthi},
Vin I 120,16. D I 3,29. Kv 13,26.

1 pl \textit{asma},
Ja II 384,5*.
\textit{asmā},
M I 334, 22.
(\textit{a})\textit{smāse},
Sn 595.
\textit{amha},
Thī 66.
\textit{mha},
Thī 66.
\textit{amhase},
Ja VI 417,13*
\textit{amhāse} / \textit{*amhāsi}
D II 275,11*.
Ja VI 553,14*.
\textit{assu},
A I 155,11.
Ja V 317,20*.

2 pl \textit{attha},
D III 81,18.

3 pl. \textit{santi},
Vin I 5,25, D I 12,29,
Vibh 339,30ff.


\item[part.pres.]
\textit{sa}(\textit{t}) / \textit{santa}
Vin I 103,11. DN II 55,19. Pp 28,21.
\textbf{med.} \textit{samāna}
Vin II 186,23. MN I 24,19. Vibh 330,14.

\item[imper.]
3 sg. \textit{atthu},
D II 22,20.
2 sg. (\textit{a})\textit{hi},
Ja VI 193,8*.

\item[opt. I]
3 sg. \textit{assa},
Vin I 4,13. D I 25,5ff. Vibh 336,34ff.
\textit{assu},
Ja IV 111,20*.
2 sg. \textit{assa},
Vin I 32,30. Ja IV 296,16*.
\textit{assasi},
M I 437,30.
2 sg. \textit{assu},
Ja VI 310,1*.
%Ja V 117,12*-13*.
1 sg. \textit{assaṃ},
Vin III 25,5ff. S IV 302,25ff.

3 pl. \textit{assu},
Dhp 74.
2 pl. \textit{assatha},
D I 3,5ff.
1 pl. \textit{assāma},
D II 307,4.
 
\item[opt. II]
3 sg. \textit{siyā},
Vin II 86,36. D I 71,33. Dhs 237,22ff.
2 sg. \textit{siyā},
Ja IV 108,28*. Sn 1119.
1 sg. \textit{siyaṃ},
M III 16,24ff. Ja VI 572,27*.

3 pl. \textit{siyuṃ}
S I 52,16* (= S[1998] 119,5*.) Ap I 21,18.
\textit{siyaṃsu}
M II 239,4.

\item[pret.]
Perf. 3 sg. \textit{āsa},
D III 155,9* (CPD Add. s. v. \textit{atthi}.)

Aor. 3 sg. \textit{āsi},
Vin I 333,34*.
S I 30,3 = S[1998] 62,3*.
Kv 443,14. 
2 sg. \textit{āsi},
Pv 139.
1 sg. \textit{āsiṃ},
Vin III 4,28.
M II 105,11*.
Ja III 274,9*.

3 pl. \textit{āsuṃ},
Vin I 229,37*.
S I 61,9*(= S[1998] 141,12*.)
\textit{āsu},
Ja IV 116,29*.
\textit{āsisuṃ},
Ap I 132,11.
\textit{āsisu},
Ap I 263,24.
1 pl. \textit{āsimhā},
Ap II 595,3*.
\textit{assumhā},
Ap I 71,15.
\end{description}

\paragraph*{Note}
\textit{atthi} is used with a pl. subject (Dhp 62; 202. Ja III 55, 4* Ap 4, 13. )
See Oberlies: Pāli Grammar §44(1); Dhp Ee (1995) note on 62.
\citet[315]{Bechert:1958} interprets it as `es gibt'.
Cf. Sadd 451, 1-3 \textit{puttā m' atthi dhanā m' atthī ti ettha atthī ti avyayapadam} (word used as indeclinable)
\textit{iva ekavacanantam pi bahuvacanantam pi bhavati}.

\textit{si}, {smi}, {mhi}: Loss of the initial \textit{a}-, because of the sandhi: Vin I 87,23 (\textit{ko 'si tvam āvuso.}) Ja II 169, 7* (\textit{naṭo smi}.)
Despite of the sandhi: M I 170, 36 (kaṃ si tvaṃ uddissa pabbajito.) Vv 1218 (tvaṃ si ācariyo mama).
Because of the law of morae: Thī 11 (\textit{mutta mhi jātimaraṇā}. )

(\textit{a})\textit{smāse}: Sn 595 \textit{kevalino 'smāse}. Pj II 463,15-16 \textit{asmāse iti amha, bhavāma.}

\textit{amhase}, \textit{amhāse} / \textit{*amhāsi}:
CPD s. v. \textit{atthi} 2(c.) For D II 275,11* \textit{*amhāsi} would be metrically better.
See further \citet[52]{alsdorf:57} for {amhāse} / \textit{*amhāsi} in Ja VI 553,14*.

(\textit{a})\textit{hi}, \textit{āhi}, \textit{astu}: Ja VI 193,8* \textit{pāṭibhogo hi kittimā ti.}
CPD (s. v. \textit{atthi}) accepts this imperativ form on the ground of \citet[36]{franke:02} and Mogg VI.53 .
As \citet[111]{Oberlies:1995} points out, \citet[35]{alsdorf:1977}, overlooking Franke's view und CPD, refers again to imper. \textit{ahi}.
Cone does not refer to this form.
On the contrary \textit{āhi} is found only in the grammar books (Sadd 450) and We find \textit{astu} only in the later texts.

%Ja IV 296,16*. (\textit{n' eva me tvaṃ pati assa}.)

\textit{assuṃ}: Used as opt. 3 pl. in Ja I 56, 30.

\textit{āsa}: See \citet[39]{vonhinueber:77}, Noman: Collected Papers VII, 99.
Otherwise \textit{āsa} is used in Ja I 451,6 and Sv I 247,28, which both explain the word \textit{itihāsa}.

1 pl. \textit{āsimhā}, Ap II 595,3*. vv. ll. \textit{āhumha}, \textit{āsumha}. Be Se \textit{ahumha}, Ce \textit{āsimha}.
%\textit{assu}: Ja V 117,12* \textit{dakkh' assu}; 13* \textit{sīlav' assu}.
%The cty explains \textit{assu} inconsistently:
%In the case of \textit{dakkh' assu}, the cty Ja V 119,30 rewords it to \textit{bhaveyyāsi} i.e. optative.
%But \textit{sīlav' assu}, the cty Ja V 119,30 rewords it to \textit{bhava} i.e. imperativ.

%\textit{assu}, Presens
%AN I 155,11 (mayam assu bho Gotama brāhmaṇā ... te c' amhā akatakalyāṇā ...)
%Ja V 317,20* (patīt' assu mayaṃ bhoto.)
%ja5/45112/ ti so hi ta.nhaamahantataaya na puurati/ siyaa ti siyu.m/ ayam eva vaa paa.tho

\textit{acchasi}, Ja VI 518,6*. Ee Be Se \textit{c' acchasi}, Ce \textit{vacchasi}.
Cf. Ja VI 519,1' \textit{acchasī ti vasissasi}. As \citet[39]{alsdorf:57}, we should read \textit{vacchasi}. $\nearrow$ \textit{vasati}.
Although CPD refers to the Burmese variant (Bd) \textit{vacchasi},
renders it rather to purely theoretical \textit{*ātsyati}.

%
% 
%12
\begin{center}
{\Large
\textbf{adeti} `eats'
}
\end{center}
\begin{description}[leftmargin=\parindent]
\item[skt.] \textit{átti} EWA \textit{AD}. CDIAL 232.
\end{description}

\begin{description}[leftmargin=\parindent]
\item[pres.] 2 sg \textit{adesi} Ja V 496, 20*.
1 sg. \textit{ademi}, Ja V 496,28*.
\textit{adāmi}, Ja VI 365,23*;24*.

3 pl. \textit{adenti} Ja VI 106,22*;26*.
\item[imper.]
3 sg. \textit{adetu}, Ja V 197, 5.
\item[opt.]
3 sg. \textit{adeyya}, Ja V 493,6.

3 pl. \textit{adeyyuṃ}, Ja II 183,6*.
\textit{adeyyu}, Ja V 107,9*; 21*.
\end{description}

\paragraph*{Note}
2 sg. \textit{adesi} Ja V 496, 20* cty Ja V 496,23 \textit{na adesi, mā khādī ti}.

\textit{adāmi}, Ja VI 365,23*;24*, is caused by the rhyming with \textit{vadāmi}.
Oberlies: Grammar of Epic Skt. (s. v. ad) reports that ad in the epics shifts to 1. class.
Thus \textit{adāmi} may be regarded as a pres. form derived from \textit{adati}.
However, this form is used only in the Ja VI 365,23*;24* and does not have any other form such as *\textit{adatu}.

\textit{adeyyu} Ja V 107,9*; 21* (Be \textit{adeyyuṃ}) is \textit{metri causa}.

%
% 
%13
\begin{center}
{\Large
$*\textbf{amati}$ `goes'
}
\end{center}
\begin{description}[leftmargin=\parindent]
\item[skt.] \textit{amī/iti}
\end{description}
\paragraph*{Prefix}
\begin{description}[leftmargin=\parindent]
\item[pati-] `go back' \textit{e}-verb
\item[imper.] 2 pl. \textit{paccametha} Ja II 358, 7*. 
\end{description}
\paragraph*{Note}
\textit{paccametha}, Ja II 358, 7*.
cty, Ja II 358,12' \textit{paccametha, puna upagacchatha}.

%
%
%14
\begin{center}
{\Large
\textbf{accati} `praises'
}
\end{center}
\begin{description}[leftmargin=\parindent]
\item[skt.] \textit{arcati} `sings'
\end{description}
\begin{description}[leftmargin=\parindent]
\item[ppp.] \textit{accita}, Ja VI 180,9*.\\
\end{description}

%
%
%15
\begin{center}
{\Large
\textbf{appeti} `connects, adopts, fits'
}
\end{center}
\begin{description}[leftmargin=\parindent]
\item[skt.] \textit{arpayati} caus. √\textit{r̥}
\end{description}
\begin{description}
\item[pres.] 3 pl. \textit{appenti}, Vin II 136,37; 137,1, Ja III 34,26*.
\item[imper.] 2 pl. \textit{appetha}, Ja VI 17,5* 
%(a, 1; Ee accetha; Ck abbaccetha, which Tr. interprets as abbetha; = āvuṇetha, Ct.)
%uppetha Be
%ubbetha Se
%appetha Ce 
\item[opt.]  1 sg. \textit{appeyyaṃ}, Vin I 347,4.
\item[pret.] 3 sg. \textit{appesi}, Ja IV 210,23*.
\item[ppp.] \textit{appita-}, Vin III 216,37; 217,3, Vibh 195,21.
\item[grd.] \textit{appiya-}, Kv 26,20.
\item[inf.] \textit{appetuṃ}, Vin II 137,3-4.
\end{description}
\paragraph*{Note}
\textit{arati} < skt. √\textit{r̥}, is found only in grammatical treatises; Dhātup 253 \textit{ara gamane}; Sadd 432 \textit{ara gatiyaṃ.}
However, grd. \textit{araṇīya-} `to be approached' is used in commentaries; CPD s.v. \textit{araṇīya}.

Ja III 34,26* \textit{appenti nimbasūlasmiṃ}. So read Ee (vv.ll. \textit{abbenti, apenti, accenti}), Be Ce.
Se \textit{accenti}.
%Be Ce appenti Se accenti.
Cf. Ja III 35,3' \textit{nimbasūle āvuṇanti}.

\textit{appetha}, Ja VI 17,5*. Read with Ce and CPD s.v. \textit{appeti}.
Cf. Ee \textit{accetha} (vv.ll. \textit{abbaccetha}, \textit{accatha}, \textit{upetha}.)
Be \textit{uppe-}, Se \textit{ubbe-}.
Cf. Trenckner: Pali Miscellany fn. 19, p. 64,  interprets as \textit{abbetha}.

 \textit{appiya-}, Kv 26,20 \textit{kāyaṃ appiyaṃ karitvā}.
Cf. Kv-a 24,12-13 \textit{kāyaṃ appe\-tabbaṃ, alliyāpetabbaṃ, ekībhāvaṃ upanetabbaṃ, avibhajitabbaṃ katvā.}
`taking [the two terms as applied to] body not in a separate but a cohesive sense, i.e., in one and the same sense, without distingushing.' (Kv trsl. fn.3, p.24.)
See also CPD s.v. 2\textit{appiya}.

\paragraph*{Prefix}
\begin{description}[leftmargin=\parindent]
\item[vi-] `destroy'
\item[ppp.] \textit{vyappita-}, Vibh 195,21.
\item[sam-] `consign, shoot (with an arrow)'
\item[pret.] 3 sg. \textit{samappayī}, Ja V 49,14* (-\textit{yī} m.c.)
\item[ppp.] \textit{samappita-}, Vin I 15,6, Sn 333 $\approx$ Dhp 315, Sn 985.
%Thī 451.
\end{description}
\paragraph*{Note}
\textit{samappayī}, Ja V 49,14* \textit{samappayī dukkatakammakārī ti.}
Cf. Ja V 49,27' \textit{puthunā sallena samappayi vijjhi.}
The long \textit{-yī} is \textit{metri causa} for Vaṃsasthā opening.
By reason of this passage, one could suppose that \textit{samappeti} means `shoot with an arrow' too.

\textit{samappita-}, has 3 different meanings:
(1) `consigned' Sn 333 $\approx$ Dhp 315, Thī 451,
(2) `endowed with, affected with, possessed of' (Vin I 15,6, Dhp 303, Ja V 102,7*-8*)
and (3) `shot (with an arrow)' Sn 985.
%Vin I 15,6, means `endowed with (inst. or comp.)'
%endowed with, possessed of
%Ja V 102,7*-8*
Cf. PED s.v. \textit{samappita}, where the third meaning is not recorded.\\

%Sn 985 sokasallasamappito
%Ja V 49,13-14* passāgataṃ puthusallena nāgaṃ/ samappayī dukatakammakārī ti vatvā.


%
%
%16
\begin{center}
{\Large
\textbf{arahati} `is worth, deserves'
}
\end{center}
\begin{description}[leftmargin=\parindent]
\item[ety.]
\textit{arhati}
%Skt. \textit{arghati}
(EWA  \textit{ARH}
%Verdienen,wert sein
CDIAL 691
SWTF \textit{arh})
\end{description}
\begin{description}[leftmargin=\parindent]
\item[pres.]3 sg. \textit{arahati}, Vin II 161,6. D I 90,10.
2 sg. \textit{arahasi}, D I 99,21.
1 sg. \textit{arahāmi}, A IV 61,8.
\end{description}
\paragraph*{Prefix}
\begin{description}[leftmargin=\parindent]
\item[paṭi] `is worthy of'
\item[pres] 3 sg. \textit{paccārahati}, Vin I 278,34.\\
\end{description}

%
%
%16
\begin{center}
{\Large
\textbf{asnāti} / \textbf{*añh{\={\v{a}ti}}} `eats'
}
\end{center}
\begin{description}[leftmargin=\parindent]
\item[ety.]
\textit{aśnāti}
EWA \textit{AŚ}
\end{description}
\begin{description}[leftmargin=\parindent]
\item[pres.] 3 sg. \textit{asnāti}, Ja VI 14,21*.
\textbf{med.} 1 sg. \textit{asniye}, Ja V 397,29*.
\item[imp.] 3 sg. \textit{asnātu}, Ja V 376,27*.
2 pl. \textit{asnātha}, D II 170,15.% =  (Ja I 3,6 Ee \textit{asanātha})
\item[fut.] 1 sg. \textit{asissāmi}, \textit{asissaṃ}, Sn 970 = Nd I 492-493,25-5.
\item[pres. part.] \textit{asamāna}, Ja V 59,14*. \textbf[med.] \textit{añhamāna}, Sn 239; 240. 
\item[ppp.] \textit{asita}, A III 30,2. Ja VI 555,16* (active sense.) \textit{anasita}, Vin IV 178,29 (active sense.)\\
\item[caus.] \textbf{ppp.} \textit{āsita} Ja V 70,8*.
\end{description}

\begin{description}[leftmargin=\parindent]
\item[deriv.]
\textit{asana} (\textit{vb. noun} `food',) Vin IV 87,27.
(\textit{vb. adj.} `eating') amatâ-, asitâ-, ghatâ-, mahâ-, (CPD s. v. \textit{asana}.)
%\textit{asana}Ja I 472,14* \textit{ghataasano} cty Ja I 472,19 \textit{so ghataṃ asanāti} (Be asnāti) \textit{tasmā ghatāsano}
\end{description}

\begin{description}[leftmargin=\parindent]
\item[ud] 'eat up'
\item[opt.] 3 pl.\textit{udāheyyuṃ /-bheyyuṃ}, M I 306,12ff (Ee \textit{udrabheyyuṃ}.)
\end{description}

\paragraph*{Stock Phrase}
\textit{asitapītakhāyitasāyita} M I 423,8-9 = M III 242,3-5 = Vibh 84,32-34. 
\paragraph*{Note}
\textit{asnāti}: Ja VI 14,21* (so Be Se Ce, \textit{asanāti} Ee.) So to read without insertion of vowel "\textit{a}"
1 sg. \textit{asniye}, Ja V 397,29* \textit{udakam pi nāsmiye} (so Ee Ce Se, \textit{nāsniye} Be.)
The cty (Ja V 405,11 aham pi bhuñjissāmi) explains the form as fut.,
but according to Lüders: Beobachtungen §177, this is a present form. \\
\textit{añhamāna}: The form with \textit{ñh} is eine genuine `western' form: Lüders: Beobachtungen §178.
opt. 3 pl.\textit{udāheyyuṃ /-bheyyuṃ};
udrabheyyuṃ is caused by a misreading of the Sinhalese script. See von Hinüber: Selected Papers 157-158, 

%
%
%17
\begin{center}
{\Large
\textbf{assati} `throws'
}
\end{center}
Simplex used only in commentaries and Sadd.
\begin{description}[leftmargin=\parindent]
\item[ety.]
\textit{asyati}
EWA \textit{AS}
\end{description}

\paragraph*{Prefix}
\begin{description}[leftmargin=\parindent]
\item[apa] 'throw away'
\item[ppp.] \textit{apattha-}, Dhp 149.
\item[nir] 'cast out'
\item[pres] 3 sg. \textit{nirassati}, Sn 785.
\item[ppp.] \textit{niratta-}, Sn 787.
\item[vi-pari-] 'reverse'
\item[ppp.] \textit{vipariyattha-}, Ja V 372,10*.
\textit{a-vipallattha-}, M III 105,2ff. 
\end{description}

\paragraph*{Note}
Aorist forms \textit{assi} and \textit{nirassi} are found in Sp I 136,2 = Mp IV 82,6.
\textit{vipariyattha-}: cty Ja V 372,11. \textit{vipariyatthan ti vipallatthaṃ}.
%pra-as pāsati Pj II 321, 29.
%upa as used in Mil 418,25.
% e-verb upāseti in there.
%However CPD doubts it and prefers to read upāsati because of other passages in the Mil.
% 


\bibliographystyle{jurabib}
\bibliography{DFG}

\end{document}